\documentclass[a4paper,12pt]{article}
\usepackage[a4paper,textwidth=40em,vmargin=22mm]{geometry}

\usepackage{calc}
\usepackage{amsmath,xltxtra,fontspec,xunicode}
\usepackage{titlesec}

\usepackage{paralist,enumitem}
\setdefaultleftmargin{3em}{2em}{2em}{1em}{1em}{1em}

\usepackage{datetime2}
\usepackage[hidelinks]{hyperref}
\hypersetup{
	colorlinks=false,
	pdfpagemode=FullScreen
}

\usepackage[PunctStyle=plain,RubberPunctSkip=false,CJKglue=\hskip 0pt,CJKecglue=\hskip 4pt plus 20pt]{xeCJK}
\usepackage{xeCJKfntef}
\XeTeXlinebreaklocale "zh"
\XeTeXlinebreakskip = 0pt

% =========================================
\usepackage{fancyhdr}
\usepackage{graphicx,eso-pic}

\usepackage{tabu,tcolorbox}

\setmainfont{Nimbus Roman}
\setromanfont{Nimbus Roman}
\setsansfont{Mona-Sans}
\setmonofont{JetBrains Mono NL}
\setCJKmainfont{Noto Serif CJK SC}
\setCJKromanfont{Noto Serif CJK SC}
\setCJKsansfont{Noto Sans CJK SC}
\setCJKmonofont{Noto Sans CJK SC}
% =========================================

\setlength{\parindent}{0em}
\setlength{\parskip}{0pt}
\setlength{\tabulinesep}{10pt}

\frenchspacing























\pagestyle{plain}
\newcommand{\NDLawCodeCoverpage}[7]{
	% argv: titleNE/titleEN/titleFR/titleZH/id/revDate/extra
	\begin{titlepage}
        \center
        \parskip=50pt
        \normalsize\sffamily
        NEOPARIA DEMOKRATIA KODIKAS \hfill \##5\par\vskip 5pt
        \vskip 50pt

        \LARGE\rmfamily
        Neoparia Demokratia #1\par
        Neoparia Demokratia #2\par
        #3 du Demokratia Neoparien\par
        纽巴尼亚共和国#4\par
        \vfill

        \vbox to 40pt{\normalsize\rmfamily#7\vfill}
        \normalsize\ttfamily
        #6

        \hypersetup{pdftitle={Neoparia Demokratia #1}}
    \end{titlepage}
}




\begin{document}
\NDLawCodeCoverpage{1}{2022-01-05}{Working Draft}{Suntagmos}{Constitution}{Constitution}{宪法}





\section*{Preface}
\begin{enumerate}
	\item We the people of Neoparia, in order to form a better union, in pursuit of Neoparianity, hereby incorporate our sovereign state Neoparia Demokratia.
	\item All powers of the Demokratia belong to its citizens.
	\item The Demokratia is a union of individual citizens, not a federation of smaller unions.
	\item The Government has no power beyond those granted to it specifically by this Constitution.
\end{enumerate}





\section{Structure of Government}
\begin{enumerate}[start=101]
	\item The Government has 3 supreme branches:
	      \begin{enumerate}
		      \item Autokrators (English: Autokrator): The Administrative branch.
		      \item Senatos (English: Senatos): The Legislation branch.
		      \item Epodikasterios (English: Supreme Court): The Judicial branch.
	      \end{enumerate}
	\item Each supreme branch is independent from another, and performs its own powers and duties on a stand-alone basis.
	\item The Government has 4 independent agencies:
	      \begin{enumerate}
		      \item Xrusomulia: The agency in charge of monetary policies.
		      \item Epoprokura: The agency in charge of legal prosecution.
	      \end{enumerate}
	\item The leaders of the independent agencies are nominated by the Autokrators for the final decisions by Senatos.
\end{enumerate}





\section{Autokrators}
\begin{enumerate}[start=201]
    \item The Autokrators is the supreme executive authority of Neoparia Demokratia.
	\item The Autokrators is elected through general election by all citizens of Neoparia Demokratia.
	\item Each term of the Autokrators lasts 5 years.
	\item When a new Autokrators is elected, it shall start its term of presidency since the third Monday of a Gregorian calendar year.
	\item The Autokrators may create and abolish Departments (Auktoritax) in the Government, and may instate Ministers to the departments.
	\item Senatos may make laws to handle the case in the event that the Autokrators is unable to reign the Government.
	\item The Autokrators may make diplomatic treaties with foreign countries.
	      However, any such diplomatic treaty must be ratified by Senatos before it becomes effective.
	\item The Autokrators may be impeached by Senatos via a 70 percent consensus.
\end{enumerate}





\section{Senatos}
\begin{enumerate}[start=301]
	\item Senatos is the supreme legislative authority of Neoparia Demokratia.
	\item Senatos may pass bills on a simple majority consensus.
	\item Senatos consists of various Senatorx, at least 20 and at most 500.
	\item Every 1000 of population adds 1 seat of Senators in Senatos.
	\item The Senatorx are elected through general election by all citizens of Neoparia Demokratia.
	\item When a new Autokrators starts its presidency, the new Senatorx start their terms in Senatos.
	\item Every Senators in Senatos has the same weight of vote.
	\item Every candidate for Senatos shall be elected as Senators if it receives votes of 2 percent of population.
	      If the quantity of elected candidates exceeds the given limit, only the most popular ones shall become Senatorx.
\end{enumerate}




\section{Epodikasterios}
\begin{enumerate}[start=401]
	\item Epodikasterios is the supreme judicial authority of Neoparia Demokratia.
	\item The principal leaders of Epodikasterios are 9 Epodikastix (English: Supreme Judge).
	\item The most senior Epodikastis is called the Magisteon Epodikastis (English: Magistrate Supreme Judge),
	      who is the nominal/ceremonial chairman of Epodikasterios.
	\item Epodikasterios has original jurisdiction over certain cases:
	      \begin{enumerate}
		      \item A case where the plaintiff or the defendant is a foreign ambassador.
		      \item A case where the plaintiff or the defendant is a judge.
		      \item A case between the Autokrators and Senatos.
	      \end{enumerate}
	\item Epodikasterios has appeal jurisdiction over all cases.
	\item A judgement made by Epodikasterios is the final decision.
	\item Epodikasterios may establish lower levels of courts, such as district courts and circuit courts,
	      to hear cases from localities.
\end{enumerate}





\section{Military Forces}
\begin{enumerate}[start=501]
	\item The Autokrators is the supreme commander of all military forces of Neoparia Demokratia.
	\item The Autokrators may maintain permanent military forces, such as army, navy, and air force.
	\item The power of declaring war belongs to Senatos.
	\item The Autokrators may move troops to foreign realm in emergency situations.
	      Without approval from Senatos, the Autokrators must recall such troops within 30 days since deployment.
	\item Senatos may make laws to give military personnel differentiated treatment in terms of crime and penalty.
\end{enumerate}





\section{Civil Rights}
\begin{enumerate}[start=601]
	\item Principle of presumed innocence: No person shall be guilty unless a court convicts its crime.
	\item Principle of due process: Any court trial of criminal case must hear the arguments from both the plaintiff and the defendant.
	\item Principle of silence: For any word a suspect provides with the police or the prosecutor,
	      it may not be used as evidence against its source in the court trial.
	\item Principle of privacy: No police officer shall search offices or resident houses without a search warrant from
	      a judge or a prosecutor.
\end{enumerate}



\end{document}
